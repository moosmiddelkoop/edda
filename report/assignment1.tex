% Options for packages loaded elsewhere
\PassOptionsToPackage{unicode}{hyperref}
\PassOptionsToPackage{hyphens}{url}
%
\documentclass[
  11pt,
]{article}
\usepackage{amsmath,amssymb}
\usepackage{lmodern}
\usepackage{iftex}
\ifPDFTeX
  \usepackage[T1]{fontenc}
  \usepackage[utf8]{inputenc}
  \usepackage{textcomp} % provide euro and other symbols
\else % if luatex or xetex
  \usepackage{unicode-math}
  \defaultfontfeatures{Scale=MatchLowercase}
  \defaultfontfeatures[\rmfamily]{Ligatures=TeX,Scale=1}
\fi
% Use upquote if available, for straight quotes in verbatim environments
\IfFileExists{upquote.sty}{\usepackage{upquote}}{}
\IfFileExists{microtype.sty}{% use microtype if available
  \usepackage[]{microtype}
  \UseMicrotypeSet[protrusion]{basicmath} % disable protrusion for tt fonts
}{}
\makeatletter
\@ifundefined{KOMAClassName}{% if non-KOMA class
  \IfFileExists{parskip.sty}{%
    \usepackage{parskip}
  }{% else
    \setlength{\parindent}{0pt}
    \setlength{\parskip}{6pt plus 2pt minus 1pt}}
}{% if KOMA class
  \KOMAoptions{parskip=half}}
\makeatother
\usepackage{xcolor}
\usepackage[margin=1in]{geometry}
\usepackage{color}
\usepackage{fancyvrb}
\newcommand{\VerbBar}{|}
\newcommand{\VERB}{\Verb[commandchars=\\\{\}]}
\DefineVerbatimEnvironment{Highlighting}{Verbatim}{commandchars=\\\{\}}
% Add ',fontsize=\small' for more characters per line
\usepackage{framed}
\definecolor{shadecolor}{RGB}{248,248,248}
\newenvironment{Shaded}{\begin{snugshade}}{\end{snugshade}}
\newcommand{\AlertTok}[1]{\textcolor[rgb]{0.94,0.16,0.16}{#1}}
\newcommand{\AnnotationTok}[1]{\textcolor[rgb]{0.56,0.35,0.01}{\textbf{\textit{#1}}}}
\newcommand{\AttributeTok}[1]{\textcolor[rgb]{0.77,0.63,0.00}{#1}}
\newcommand{\BaseNTok}[1]{\textcolor[rgb]{0.00,0.00,0.81}{#1}}
\newcommand{\BuiltInTok}[1]{#1}
\newcommand{\CharTok}[1]{\textcolor[rgb]{0.31,0.60,0.02}{#1}}
\newcommand{\CommentTok}[1]{\textcolor[rgb]{0.56,0.35,0.01}{\textit{#1}}}
\newcommand{\CommentVarTok}[1]{\textcolor[rgb]{0.56,0.35,0.01}{\textbf{\textit{#1}}}}
\newcommand{\ConstantTok}[1]{\textcolor[rgb]{0.00,0.00,0.00}{#1}}
\newcommand{\ControlFlowTok}[1]{\textcolor[rgb]{0.13,0.29,0.53}{\textbf{#1}}}
\newcommand{\DataTypeTok}[1]{\textcolor[rgb]{0.13,0.29,0.53}{#1}}
\newcommand{\DecValTok}[1]{\textcolor[rgb]{0.00,0.00,0.81}{#1}}
\newcommand{\DocumentationTok}[1]{\textcolor[rgb]{0.56,0.35,0.01}{\textbf{\textit{#1}}}}
\newcommand{\ErrorTok}[1]{\textcolor[rgb]{0.64,0.00,0.00}{\textbf{#1}}}
\newcommand{\ExtensionTok}[1]{#1}
\newcommand{\FloatTok}[1]{\textcolor[rgb]{0.00,0.00,0.81}{#1}}
\newcommand{\FunctionTok}[1]{\textcolor[rgb]{0.00,0.00,0.00}{#1}}
\newcommand{\ImportTok}[1]{#1}
\newcommand{\InformationTok}[1]{\textcolor[rgb]{0.56,0.35,0.01}{\textbf{\textit{#1}}}}
\newcommand{\KeywordTok}[1]{\textcolor[rgb]{0.13,0.29,0.53}{\textbf{#1}}}
\newcommand{\NormalTok}[1]{#1}
\newcommand{\OperatorTok}[1]{\textcolor[rgb]{0.81,0.36,0.00}{\textbf{#1}}}
\newcommand{\OtherTok}[1]{\textcolor[rgb]{0.56,0.35,0.01}{#1}}
\newcommand{\PreprocessorTok}[1]{\textcolor[rgb]{0.56,0.35,0.01}{\textit{#1}}}
\newcommand{\RegionMarkerTok}[1]{#1}
\newcommand{\SpecialCharTok}[1]{\textcolor[rgb]{0.00,0.00,0.00}{#1}}
\newcommand{\SpecialStringTok}[1]{\textcolor[rgb]{0.31,0.60,0.02}{#1}}
\newcommand{\StringTok}[1]{\textcolor[rgb]{0.31,0.60,0.02}{#1}}
\newcommand{\VariableTok}[1]{\textcolor[rgb]{0.00,0.00,0.00}{#1}}
\newcommand{\VerbatimStringTok}[1]{\textcolor[rgb]{0.31,0.60,0.02}{#1}}
\newcommand{\WarningTok}[1]{\textcolor[rgb]{0.56,0.35,0.01}{\textbf{\textit{#1}}}}
\usepackage{longtable,booktabs,array}
\usepackage{calc} % for calculating minipage widths
% Correct order of tables after \paragraph or \subparagraph
\usepackage{etoolbox}
\makeatletter
\patchcmd\longtable{\par}{\if@noskipsec\mbox{}\fi\par}{}{}
\makeatother
% Allow footnotes in longtable head/foot
\IfFileExists{footnotehyper.sty}{\usepackage{footnotehyper}}{\usepackage{footnote}}
\makesavenoteenv{longtable}
\usepackage{graphicx}
\makeatletter
\def\maxwidth{\ifdim\Gin@nat@width>\linewidth\linewidth\else\Gin@nat@width\fi}
\def\maxheight{\ifdim\Gin@nat@height>\textheight\textheight\else\Gin@nat@height\fi}
\makeatother
% Scale images if necessary, so that they will not overflow the page
% margins by default, and it is still possible to overwrite the defaults
% using explicit options in \includegraphics[width, height, ...]{}
\setkeys{Gin}{width=\maxwidth,height=\maxheight,keepaspectratio}
% Set default figure placement to htbp
\makeatletter
\def\fps@figure{htbp}
\makeatother
\setlength{\emergencystretch}{3em} % prevent overfull lines
\providecommand{\tightlist}{%
  \setlength{\itemsep}{0pt}\setlength{\parskip}{0pt}}
\setcounter{secnumdepth}{-\maxdimen} % remove section numbering
\ifLuaTeX
  \usepackage{selnolig}  % disable illegal ligatures
\fi
\IfFileExists{bookmark.sty}{\usepackage{bookmark}}{\usepackage{hyperref}}
\IfFileExists{xurl.sty}{\usepackage{xurl}}{} % add URL line breaks if available
\urlstyle{same} % disable monospaced font for URLs
\hypersetup{
  pdftitle={Assignment 1},
  pdfauthor={Gijs Smeets, group 35},
  hidelinks,
  pdfcreator={LaTeX via pandoc}}

\title{Assignment 1}
\author{Gijs Smeets, group 35}
\date{February 2023}

\begin{document}
\maketitle

\hypertarget{short-introduction-to-r-markdown}{%
\subsection{Short introduction to R
Markdown}\label{short-introduction-to-r-markdown}}

This is an R Markdown document. Markdown is a simple formatting syntax
for authoring HTML, PDF, and MS Word documents. R Markdown files permit
you to interweave R code with ordinary text to produce well-formatted
data analysis reports that are easy to modify. The R Markdown file
itself shows the readers exactly how you got the results in your report.
For more details on using R Markdown see
\href{http://rmarkdown.rstudio.com}{\color{blue}{\underline{http://rmarkdown.rstudio.com}}}.

When you click the \textbf{Knit} button, a document will be generated
that includes both content as well as the output of any embedded R code
chunks within the document. For inline R code, surround code with back
ticks and r. R replaces inline code with its results. For example, two
plus one is 3; for the build-in R dataset \texttt{cars}, there were 50
cars studied. You can embed an R code chunk like this:

\begin{Shaded}
\begin{Highlighting}[]
\FunctionTok{summary}\NormalTok{(cars)}
\end{Highlighting}
\end{Shaded}

\begin{verbatim}
##      speed           dist       
##  Min.   : 4.0   Min.   :  2.00  
##  1st Qu.:12.0   1st Qu.: 26.00  
##  Median :15.0   Median : 36.00  
##  Mean   :15.4   Mean   : 42.98  
##  3rd Qu.:19.0   3rd Qu.: 56.00  
##  Max.   :25.0   Max.   :120.00
\end{verbatim}

\hypertarget{figures}{%
\subsubsection{Figures}\label{figures}}

You can also embed plots, for example:

\includegraphics{assignment1_files/figure-latex/unnamed-chunk-2-1.pdf}

Note that the \texttt{echo\ =\ FALSE} parameter was added to the code
chunk to prevent printing of the R code that generated the plot. Use
knitr options to style the output of a chunk. Place options in brackets
above the chunk. Other options with the defaults are: the
\texttt{eval=FALSE} option just displays the R code (and does not run
it); \texttt{warning=TRUE} whether to display warnings;
\texttt{tidy=TRUE} wraps long code so it does not run off the page.

You can control the size and placement of figures. For example, you can
put two figures (or more) next to each other. Use
\texttt{par(mfrow=c(n,m))} to create \texttt{n} by \texttt{m} plots in
one picture in R. You can adjust the proportions of figures by using the
\texttt{fig.width} and \texttt{fig.height} chunk options. These are
specified in inches, and will be automatically scaled down to fit within
the handout margin. Chunk option \texttt{fig.align} takes values
\texttt{left}, \texttt{right}, or \texttt{center} (to align figures in
the output document).

\begin{Shaded}
\begin{Highlighting}[]
\FunctionTok{par}\NormalTok{(}\AttributeTok{mfrow=}\FunctionTok{c}\NormalTok{(}\DecValTok{1}\NormalTok{,}\DecValTok{2}\NormalTok{)); x1}\OtherTok{=}\FunctionTok{rnorm}\NormalTok{(}\DecValTok{50}\NormalTok{); }\FunctionTok{hist}\NormalTok{(x1); }\FunctionTok{qqnorm}\NormalTok{(x1)}
\end{Highlighting}
\end{Shaded}

\begin{center}\includegraphics{assignment1_files/figure-latex/unnamed-chunk-3-1} \end{center}

You can arrange for figures to span across the entire page by using the
\texttt{fig.fullwidth} chunk option.

\begin{Shaded}
\begin{Highlighting}[]
\FunctionTok{plot}\NormalTok{(iris}\SpecialCharTok{$}\NormalTok{Sepal.Length,iris}\SpecialCharTok{$}\NormalTok{Petal.Length,}\AttributeTok{xlab=}\StringTok{"Sepal.Length"}\NormalTok{,}\AttributeTok{ylab=}\StringTok{"Petal.Length"}\NormalTok{)}
\end{Highlighting}
\end{Shaded}

\includegraphics{assignment1_files/figure-latex/unnamed-chunk-4-1.pdf}

More about chunk options can be found at
\href{https://yihui.name/knitr/options/}{\color{blue}{\underline{https://yihui.name/knitr/options/}}}.

\hypertarget{equations}{%
\subsubsection{Equations}\label{equations}}

To produce mathematical symbols, you can also include
\LaTeX~expessions/equations in your report: inline
\(\frac{d}{dx}\left(\int_{0}^{x} f(u)\,du\right)=f(x)\) and in the
display mode: \[
\frac{d}{dx}\left( \int_{0}^{x} f(u)\,du\right)=f(x).
\] To be able to use this functionality, \LaTeX~has to be installed.

\hypertarget{footnotes}{%
\subsubsection{Footnotes}\label{footnotes}}

Here is the use of a footnote\footnote{This is a footnote.}.

\hypertarget{images}{%
\subsubsection{Images}\label{images}}

Want an image? This will do it. To depict an image (say,
\texttt{my\_image.jpg} which should be in your current working
directory), use this command

\begin{figure}
\centering
\includegraphics{my_image.jpg}
\caption{caption for my image}
\end{figure}

\hypertarget{tables}{%
\subsubsection{Tables}\label{tables}}

Want a table? This will create one (note that the separators \emph{do
not} have to be aligned).

\begin{longtable}[]{@{}ll@{}}
\toprule()
Table Header & Second Header \\
\midrule()
\endhead
Table Cell & Cell 2 \\
Cell 3 & Cell 4 \\
\bottomrule()
\end{longtable}

You can also make table by using knit's \texttt{kable} function:

\begin{longtable}[]{@{}
  >{\raggedright\arraybackslash}p{(\columnwidth - 22\tabcolsep) * \real{0.2609}}
  >{\raggedleft\arraybackslash}p{(\columnwidth - 22\tabcolsep) * \real{0.0725}}
  >{\raggedleft\arraybackslash}p{(\columnwidth - 22\tabcolsep) * \real{0.0580}}
  >{\raggedleft\arraybackslash}p{(\columnwidth - 22\tabcolsep) * \real{0.0725}}
  >{\raggedleft\arraybackslash}p{(\columnwidth - 22\tabcolsep) * \real{0.0580}}
  >{\raggedleft\arraybackslash}p{(\columnwidth - 22\tabcolsep) * \real{0.0725}}
  >{\raggedleft\arraybackslash}p{(\columnwidth - 22\tabcolsep) * \real{0.0870}}
  >{\raggedleft\arraybackslash}p{(\columnwidth - 22\tabcolsep) * \real{0.0870}}
  >{\raggedleft\arraybackslash}p{(\columnwidth - 22\tabcolsep) * \real{0.0435}}
  >{\raggedleft\arraybackslash}p{(\columnwidth - 22\tabcolsep) * \real{0.0435}}
  >{\raggedleft\arraybackslash}p{(\columnwidth - 22\tabcolsep) * \real{0.0725}}
  >{\raggedleft\arraybackslash}p{(\columnwidth - 22\tabcolsep) * \real{0.0725}}@{}}
\caption{A knit kable.}\tabularnewline
\toprule()
\begin{minipage}[b]{\linewidth}\raggedright
\end{minipage} & \begin{minipage}[b]{\linewidth}\raggedleft
mpg
\end{minipage} & \begin{minipage}[b]{\linewidth}\raggedleft
cyl
\end{minipage} & \begin{minipage}[b]{\linewidth}\raggedleft
disp
\end{minipage} & \begin{minipage}[b]{\linewidth}\raggedleft
hp
\end{minipage} & \begin{minipage}[b]{\linewidth}\raggedleft
drat
\end{minipage} & \begin{minipage}[b]{\linewidth}\raggedleft
wt
\end{minipage} & \begin{minipage}[b]{\linewidth}\raggedleft
qsec
\end{minipage} & \begin{minipage}[b]{\linewidth}\raggedleft
vs
\end{minipage} & \begin{minipage}[b]{\linewidth}\raggedleft
am
\end{minipage} & \begin{minipage}[b]{\linewidth}\raggedleft
gear
\end{minipage} & \begin{minipage}[b]{\linewidth}\raggedleft
carb
\end{minipage} \\
\midrule()
\endfirsthead
\toprule()
\begin{minipage}[b]{\linewidth}\raggedright
\end{minipage} & \begin{minipage}[b]{\linewidth}\raggedleft
mpg
\end{minipage} & \begin{minipage}[b]{\linewidth}\raggedleft
cyl
\end{minipage} & \begin{minipage}[b]{\linewidth}\raggedleft
disp
\end{minipage} & \begin{minipage}[b]{\linewidth}\raggedleft
hp
\end{minipage} & \begin{minipage}[b]{\linewidth}\raggedleft
drat
\end{minipage} & \begin{minipage}[b]{\linewidth}\raggedleft
wt
\end{minipage} & \begin{minipage}[b]{\linewidth}\raggedleft
qsec
\end{minipage} & \begin{minipage}[b]{\linewidth}\raggedleft
vs
\end{minipage} & \begin{minipage}[b]{\linewidth}\raggedleft
am
\end{minipage} & \begin{minipage}[b]{\linewidth}\raggedleft
gear
\end{minipage} & \begin{minipage}[b]{\linewidth}\raggedleft
carb
\end{minipage} \\
\midrule()
\endhead
Mazda RX4 & 21.0 & 6 & 160 & 110 & 3.90 & 2.620 & 16.46 & 0 & 1 & 4 &
4 \\
Mazda RX4 Wag & 21.0 & 6 & 160 & 110 & 3.90 & 2.875 & 17.02 & 0 & 1 & 4
& 4 \\
Datsun 710 & 22.8 & 4 & 108 & 93 & 3.85 & 2.320 & 18.61 & 1 & 1 & 4 &
1 \\
Hornet 4 Drive & 21.4 & 6 & 258 & 110 & 3.08 & 3.215 & 19.44 & 1 & 0 & 3
& 1 \\
Hornet Sportabout & 18.7 & 8 & 360 & 175 & 3.15 & 3.440 & 17.02 & 0 & 0
& 3 & 2 \\
\bottomrule()
\end{longtable}

\hypertarget{block-quote}{%
\subsubsection{Block quote}\label{block-quote}}

\begin{quote}
This will create a block quote, if you want one.
\end{quote}

\hypertarget{verbatim}{%
\subsubsection{Verbatim}\label{verbatim}}

\begin{verbatim}
This text is displayed verbatim/preformatted.
\end{verbatim}

\hypertarget{links}{%
\subsubsection{Links}\label{links}}

Links: \url{http://example.com}, \href{http://google.com}{in-text link
to Google}.

This is a \hyperlink{target1}{{\color{blue}{\underline{hyperlink}}}}.

\hypertarget{target1}{{\color{blue}{\underline{This}}}}

is where the hyperlink jumps to.

\hypertarget{itimization-italicized-and-embolded-text}{%
\subsubsection{Itimization, italicized and embolded
text}\label{itimization-italicized-and-embolded-text}}

\begin{itemize}
\tightlist
\item
  Single asterisks italicize text \emph{like this}.
\item
  Double asterisks embolden text \textbf{like this}.
\end{itemize}

One more way to italicize and embold: \emph{italic} and \textbf{bold}.

\hypertarget{exercise-1}{%
\subsection{Exercise 1}\label{exercise-1}}

Below is a template for reporting the exercises from the assignments.

\textbf{a)} Here are some consequitive R-commands.

\begin{Shaded}
\begin{Highlighting}[]
\NormalTok{x}\OtherTok{=}\FunctionTok{rep}\NormalTok{(}\FunctionTok{c}\NormalTok{(}\StringTok{"A"}\NormalTok{,}\StringTok{"B"}\NormalTok{),}\AttributeTok{each=}\DecValTok{5}\NormalTok{); x}
\end{Highlighting}
\end{Shaded}

\begin{verbatim}
##  [1] "A" "A" "A" "A" "A" "B" "B" "B" "B" "B"
\end{verbatim}

\begin{Shaded}
\begin{Highlighting}[]
\FunctionTok{sample}\NormalTok{(x)}
\end{Highlighting}
\end{Shaded}

\begin{verbatim}
##  [1] "B" "A" "A" "B" "B" "A" "A" "B" "B" "A"
\end{verbatim}

\begin{Shaded}
\begin{Highlighting}[]
\NormalTok{x}\OtherTok{=}\FunctionTok{rnorm}\NormalTok{(}\DecValTok{100}\NormalTok{)}
\end{Highlighting}
\end{Shaded}

Now the same code chunk but with all the output collapsed into signle
block.

\begin{Shaded}
\begin{Highlighting}[]
\NormalTok{x}\OtherTok{=}\FunctionTok{rep}\NormalTok{(}\FunctionTok{c}\NormalTok{(}\StringTok{"A"}\NormalTok{,}\StringTok{"B"}\NormalTok{),}\AttributeTok{each=}\DecValTok{5}\NormalTok{); x}
\DocumentationTok{\#\#  [1] "A" "A" "A" "A" "A" "B" "B" "B" "B" "B"}
\FunctionTok{sample}\NormalTok{(x)}
\DocumentationTok{\#\#  [1] "A" "B" "A" "B" "B" "A" "A" "B" "B" "A"}
\NormalTok{x}\OtherTok{=}\FunctionTok{rnorm}\NormalTok{(}\DecValTok{100}\NormalTok{)}
\end{Highlighting}
\end{Shaded}

\textbf{b)} Below we perform a one sample t-test for the artificial data
(that we generate ourselves).

\begin{Shaded}
\begin{Highlighting}[]
\NormalTok{mu}\OtherTok{=}\FloatTok{0.2}
\NormalTok{x}\OtherTok{=}\FunctionTok{rnorm}\NormalTok{(}\DecValTok{100}\NormalTok{,mu,}\DecValTok{1}\NormalTok{) }\CommentTok{\# creating artificial data}
\FunctionTok{t.test}\NormalTok{(x,}\AttributeTok{mean=}\DecValTok{0}\NormalTok{)   }\CommentTok{\# t.test(x,alternative=c("two.sided"),conf.level=0.95,mu=10)}
\end{Highlighting}
\end{Shaded}

\begin{verbatim}
## 
##  One Sample t-test
## 
## data:  x
## t = 1.4762, df = 99, p-value = 0.1431
## alternative hypothesis: true mean is not equal to 0
## 95 percent confidence interval:
##  -0.05235667  0.35665932
## sample estimates:
## mean of x 
## 0.1521513
\end{verbatim}

\textbf{c)} We often do not need to report the whole output of
R-commands, only certain values of the output. For example, below we
perform a two-sample t-test and report only the (appropriately rounded)
values of t-statistics and the p-pavue.

\begin{Shaded}
\begin{Highlighting}[]
\NormalTok{mu}\OtherTok{=}\DecValTok{0}\NormalTok{;nu}\OtherTok{=}\FloatTok{0.5}
\NormalTok{x}\OtherTok{=}\FunctionTok{rnorm}\NormalTok{(}\DecValTok{50}\NormalTok{,mu,}\DecValTok{1}\NormalTok{); y}\OtherTok{=}\FunctionTok{rnorm}\NormalTok{(}\DecValTok{50}\NormalTok{,nu,}\DecValTok{1}\NormalTok{) }\CommentTok{\# creating artificial data}
\NormalTok{ttest}\OtherTok{=}\FunctionTok{t.test}\NormalTok{(x,y) }
\end{Highlighting}
\end{Shaded}

The value of t-statistics in the above evaluation is -4.03 and the
p-value is \ensuremath{10^{-4}}.

\end{document}
